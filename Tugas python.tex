\documentclass{article}
\usepackage{listings}
\usepackage{color}
\usepackage{sectsty}
\usepackage[parfill]{parskip}

\definecolor{dkgreen}{rgb}{0,0.6,0}
\definecolor{gray}{rgb}{0.5,0.5,0.5}
\definecolor{mauve}{rgb}{0.58,0,0.82}

\lstset{frame=tb,
	language=Python,
	aboveskip=1mm,
	belowskip=3mm,
	showstringspaces=false,
	columns=flexible,
	basicstyle={\small\ttfamily},
	numbers=none,
	numberstyle=\tiny\color{gray},
	keywordstyle=\color{blue},
	commentstyle=\color{dkgreen},
	stringstyle=\color{mauve},
	breaklines=true,
	breakatwhitespace=true,
	tabsize=3
}

\sectionfont{\fontsize{12}{12}\selectfont}

\begin{document}
Nama	: Muhammad Nanda Fahriza \\
NPM		: 1194057 \\
Kelas	: D4TI-3B \\

\section{Menambah key baru pada dictionary}
\begin{lstlisting}
	d = {'key': 'value'}
	print(d) # {'key': 'value'}
	
	d['mynewkey'] = 'mynewvalue'
	
	print(d) # {'key': 'value',
	'mynewkey': 'mynewvalue'}
\end{lstlisting}

\section{Cara memilih item secara acak dari daftar}
\begin{lstlisting}
	import random
		
	foo = ['battery', 'correct', 'horse', 'staple']
	secure_random = random.SystemRandom()
	print(secure_random.choice(foo))
\end{lstlisting}

\section{Mengimpor file dari folder berbeda}
\begin{lstlisting}
	import os, sys
		
	from os.path import dirname, join, abspath
	sys.path.insert(0, abspath(join(dirname(__file__), '..')))
	
	from root_folder import file_name
\end{lstlisting}

\section{Membaca JSON memiliki banyak items}
\begin{lstlisting}
	data JSON dari API jsonplaceholder dengan endpoint URL: https://jsonplaceholder.typicode.com/posts
	
	Buatlah program baru dengan nama list_artikel.py, kemudian isi dengan kode berikut:
	
	import json
	from urllib import request
	
	url = "https://jsonplaceholder.typicode.com/posts"
	
	#lakukan http request
	response = request.urlopen(url)
	# parsing data json
	data = json.loads(response.read())
	#gunakan perulangan untuk menampilkan data
	for i in range(len(data)):
	print(f"{i}. {data[i]['title']}")
\end{lstlisting}

\section{Cara menampilkan Input dari Keyboard}
\begin{lstlisting}
	Python sudah menyediakan fungsi input() dan raw_input() untuk mengambil inputan dari keyboard.
	
	Cara pakainya:
	
	nama_varabel = input("Sebuah Teks")
	
	# Mengambil input
	nama = raw_input("Siapa nama kamu: ")
	umur = input("Berapa umur kamu: ")
	
	# Menampilkan output
	print "Hello",nama,"umur kamu adalah",umur,"tahun"
\end{lstlisting}

\section{Cara menampilkan output}
\begin{lstlisting}
	nama = "Muhammad Nanda Fahriza"
	print "Hello",nama
\end{lstlisting}

\section{Menggunakan string Formatting}
\begin{lstlisting}
	nama = raw_input("Inputkan nama: ")
	umur = input("Inputkan umur: ")
	tinggi = input("Inputkan tinggi badan: ")
	
	print "Hello %s, saat ini usiamu %d tahun dan tinggi badanmu %f cm" % (nama, umur, tinggi)
\end{lstlisting}

\section{Menggunakan Perulangan for}
\begin{lstlisting}
	item = ['kopi','nasi','teh','jeruk']
	
	for isi in item:
	print(isi)
\end{lstlisting}

\section{Menggunakan Perulangan while}
\begin{lstlisting}
	jawab = 'ya'
	hitung = 0
	
	while(jawab == 'ya'):
	hitung += 1
	jawab = input("Ulang lagi tidak? ")
	
	print(f"Total perulagan: {hitung}")
\end{lstlisting}

\section{Cara pakai operator relasi sama dengan}
\begin{lstlisting}
	lulus = raw_input("Apakah kamu lulus? [ya/tidak]: ")
	
	if lulus == "tidak":
	print("Kamu harus ikut ujian")
\end{lstlisting}

\section{Penggunaan If/Else}
\begin{lstlisting}
	umur = input("Berapa umur kamu: ")
	
	if umur >= 18:
	print("Kamu boleh membuat SIM")
	else:
	print("Kamu belum boleh membuat SIM")
\end{lstlisting}

\section{Membuat Program Dengan list}
\begin{lstlisting}
	# Membuat list kosong untuk menampung hobi
	hobi = []
	stop = False
	i = 0
	
	# Mengisi hobi
	while(not stop):
	hobi_baru = raw_input("Inputkan hobi yang ke-{}: ".format(i))
	hobi.append(hobi_baru)
	
	# Increment i
	i += 1
	
	tanya = raw_input("Mau isi lagi? (y/t): ")
	if(tanya == "t"): 
	stop = True
	
	# Cetak Semua Hobi
	print "=" * 10 
	print "Kamu memiliki {} hobi".format(len(hobi))
	for hb in hobi:
	print "- {}".format(hb)
\end{lstlisting}

\section{Pengambilan panjang Tuple}
\begin{lstlisting}
	# Membuat Tuple
	hari = ('Senin', 'Selasa', 'Rabu', 'Kamis', 'Jum\'at', 'Sabtu', 'Minggu')
	
	# Mengambil panjang tuple hari
	print("Jumlah hari: %d" % len(hari))
\end{lstlisting}

\section{Mengakses nilai ietm dari dictionary}
\begin{lstlisting}
	# Membuat Dictionary
	mahasiswa = {
		"nama": "Muhammad Nanda",
		"umur": 22,
		"hobi": ["coding", "membaca", "belajar"],
		"menikah": False,
		"sosmed": {
			"facebook": "muhammadnandafahriza",
			"twitter": "@nandafahriza"
		} 
	}
	
	# Mengakses isi dictionary
	print("Nama saya adalah %s" % mahasiswa["nama"])
	print("Twitter: %s" % mahasiswa["sosmed"]["twitter"])
\end{lstlisting}

\section{Membuat fungsi mengembalikan nilai}
\begin{lstlisting}
	def luas_persegi(sisi):
	luas = sisi * sisi
	return luas
	
	# pemanggilan fungsi
	print "Luas persegi: %d" % luas_persegi(6)
\end{lstlisting}

\section{Cara menggunakan lambda expression}
\begin{lstlisting}
	greeting = lambda name: print(f"Hello, {name}") 
\end{lstlisting}

\section{Penggunaan dari *args dan **kwargs}
\begin{lstlisting}
	# membuat fungsi dengan parameter *args
	def kirim_sms(*nomer):
	print nomer
	
	# membuat fungsi dengan parameter **kwargs
	def tulis_sms(**isi):
	print isi
	
	# Pemanggilan fungsi *args
	kirim_sms(123, 888, 4444)
	
	# pemanggilan fungsi **kwargs
	tulis_sms(tujuan=123, pesan="apa kabar")
\end{lstlisting}

\section{Parsing XML di Python}
\begin{lstlisting}
	import xml.dom.minidom as minidom
	
	def main():
	# gunakan fungsi parse() untuk me-load xml ke memori 
	# dan melakukan parsing
	doc = minidom.parse("mahasiswa.xml")
	
	# Cetak isi doc dan tag pertamanya
	print doc.nodeName
	print doc.firstChild.tagName
	
	
	if __name__ == "__main__":
	main()
	
\end{lstlisting}

\section{Cara baca dan Parse file CSV di Python}
\begin{lstlisting}
	import csv
	
	with open('contacts.csv') as csv_file:
	csv_reader = csv.reader(csv_file, delimiter=",")
	print(csv_reader)
	for row in csv_reader:
	print(row)
\end{lstlisting}

\section{ Menulis data dictionary ke CSV}
\begin{lstlisting}
	import csv
	
	with open('contacts.csv', mode='a') as csv_file:
	# menentukan label
	fieldnames = ['NO', 'NAMA', 'TELEPON']
	
	# membuat objek writer
	writer = csv.DictWriter(csv_file, fieldnames=fieldnames)
	
	# menulis baris ke file CSV
	writer.writeheader()
	writer.writerow({'NO': '10', 'NAMA': 'Nanda', 'TELEPON': '02109999'})
	writer.writerow({'NO': '11', 'NAMA': 'Fahriza', 'TELEPON': '02148488888'})
	
	print("Writing Done!")
\end{lstlisting}

\section{Membuat fungsi lihat data}
\begin{lstlisting}
	def show_contact():
	clear_screen()
	contacts = []
	with open(csv_filename) as csv_file:
	csv_reader = csv.reader(csv_file, delimiter=",")
	for row in csv_reader:
	contacts.append(row)
	
	if (len(contacts) > 0):
	labels = contacts.pop(0)
	print(f"{labels[0]} \t {labels[1]} \t\t {labels[2]}")
	print("-"*34)
	for data in contacts:
	print(f'{data[0]} \t {data[1]} \t {data[2]}')
	else:
	print("Tidak ada data!")
	back_to_menu()
	
\end{lstlisting}

\section{Menggunakan fungsi create}
\begin{lstlisting}
	def create_contact():
	clear_screen()
	with open(csv_filename, mode='a') as csv_file:
	fieldnames = ['NO', 'NAMA', 'TELEPON']
	writer = csv.DictWriter(csv_file, fieldnames=fieldnames)
	
	no = input("No urut: ")
	nama = input("Nama lengkap: ")
	telepon = input("No. Telepon: ")
	
	writer.writerow({'NO': no, 'NAMA': nama, 'TELEPON': telepon})    
	print("Berhasil disimpan!")
	
	back_to_menu()
\end{lstlisting}

\section{Membuat fungsi search}
\begin{lstlisting}
	def search_contact():
	clear_screen()
	contacts = []
	
	with open(csv_filename, mode="r") as csv_file:
	csv_reader = csv.DictReader(csv_file)
	for row in csv_reader:
	contacts.append(row)
	
	no = input("Cari berdasrakan nomer urut> ")
	
	data_found = []
	
	# mencari contact
	indeks = 0
	for data in contacts:
	if (data['NO'] == no):
	data_found = contacts[indeks]
	
	indeks = indeks + 1
	
	if len(data_found) > 0:
	print("DATA DITEMUKAN: ")
	print(f"Nama: {data_found['NAMA']}")
	print(f"Telepon: {data_found['TELEPON']}")
	else:
	print("Tidak ada data ditemukan")
	back_to_menu()
\end{lstlisting}

\section{Membuat fungsi edit}
\begin{lstlisting}
	def edit_contact():
	clear_screen()
	contacts = []
	
	with open(csv_filename, mode="r") as csv_file:
	csv_reader = csv.DictReader(csv_file)
	for row in csv_reader:
	contacts.append(row)
	
	print("NO \t NAMA \t\t TELEPON")
	print("-" * 32)
	
	for data in contacts:
	print(f"{data['NO']} \t {data['NAMA']} \t {data['TELEPON']}")
	
	print("-----------------------")
	no = input("Pilih nomer kontak> ")
	nama = input("nama baru: ")
	telepon = input("nomer telepon baru: ")
	
	# mencari contact dan mengubah datanya
	# dengan data yang baru
	indeks = 0
	for data in contacts:
	if (data['NO'] == no):
	contacts[indeks]['NAMA'] = nama
	contacts[indeks]['TELEPON'] = telepon
	indeks = indeks + 1
	
	# Menulis data baru ke file CSV (tulis ulang)
	with open(csv_filename, mode="w") as csv_file:
	fieldnames = ['NO', 'NAMA', 'TELEPON']
	writer = csv.DictWriter(csv_file, fieldnames=fieldnames)
	writer.writeheader()
	for new_data in contacts:
	writer.writerow({'NO': new_data['NO'], 'NAMA': new_data['NAMA'], 'TELEPON': new_data['TELEPON']}) 
	
	back_to_menu()
\end{lstlisting}

\section{Menggunakan fungsi delete}
\begin{lstlisting}
	def delete_contact():
	clear_screen()
	contacts = []
	
	with open(csv_filename, mode="r") as csv_file:
	csv_reader = csv.DictReader(csv_file)
	for row in csv_reader:
	contacts.append(row)
	
	print("NO \t NAMA \t\t TELEPON")
	print("-" * 32)
	
	for data in contacts:
	print(f"{data['NO']} \t {data['NAMA']} \t {data['TELEPON']}")
	
	print("-----------------------")
	no = input("Hapus nomer> ")
	
	# mencari contact dan mengubah datanya
	# dengan data yang baru
	indeks = 0
	for data in contacts:
	if (data['NO'] == no):
	contacts.remove(contacts[indeks])
	indeks = indeks + 1
	
	# Menulis data baru ke file CSV (tulis ulang)
	with open(csv_filename, mode="w") as csv_file:
	fieldnames = ['NO', 'NAMA', 'TELEPON']
	writer = csv.DictWriter(csv_file, fieldnames=fieldnames)
	writer.writeheader()
	for new_data in contacts:
	writer.writerow({'NO': new_data['NO'], 'NAMA': new_data['NAMA'], 'TELEPON': new_data['TELEPON']}) 
	
	print("Data sudah terhapus")
	back_to_menu()
\end{lstlisting}

\section{Menggunakan main loop}
\begin{lstlisting}
	if __name__ == "__main__":
	while True:
	show_menu()
\end{lstlisting}

\section{Menangani eksepsi}
\begin{lstlisting}
	# import modul sys untuk memperoleh jenis eksepsi
	import sys
	
	lists = ['a', 0, 4]
	for each in lists:
	try:
	print("Masukan:", each)
	r = 1/int(each)
	break
	except:
	print("Upps!", sys.exc_info()[0], " terjadi.")
	print("Masukan berikutnya.")
	print()
	print("Kebalikan dari ", each, " =", r)
\end{lstlisting}

	\section {UnicodeEncodeError: 'ascii' codec can't encode character u'\textbackslash xa0' in position 20: ordinal not in range(128)}
Pada dasarnya, berhenti menggunakan str untuk mengonversi dari unicode ke teks/byte yang diencode. \\
Sebagai gantinya, gunakan .encode() dengan benar untuk mengencode string:
\begin{lstlisting}
	p.agent_info = u' '.join((agent_contact, agent_telno)).encode('utf-8').strip()
\end{lstlisting}

\section {Memastikan apakah direktori ada di Python}
\begin{lstlisting}
	>>> import os
	>>> os.path.isdir('new_folder')
	True
	>>> os.path.exists(os.path.join(os.getcwd(), 'new_folder', 'file.txt'))
	False
\end{lstlisting}

\section {Cara merujuk ke objek null pada Python}
\begin{lstlisting}
	if foo is None:
\end{lstlisting}

\section {Cara cek nilai NaN pada Python}
\begin{lstlisting}
	>>> import math
	>>> x = float('nan')
	>>> math.isnan(x)
	True
\end{lstlisting}

\section {Menghapus duplikasi pada list}
\begin{lstlisting}
	>>> t = [1, 2, 3, 1, 2, 5, 6, 7, 8]
	>>> t
	[1, 2, 3, 1, 2, 5, 6, 7, 8]
	>>> list(set(t))
	[1, 2, 3, 5, 6, 7, 8]
	>>> s = [1, 2, 3]
	>>> list(set(t) - set(s))
	[8, 5, 6, 7]
\end{lstlisting}

\section {Sort dictionary berdasarkan key}
\begin{lstlisting}
	In [1]: import collections
	
	In [2]: d = {2:3, 1:89, 4:5, 3:0}
	
	In [3]: od = collections.OrderedDict(sorted(d.items()))
	
	In [4]: od
	Out[4]: OrderedDict([(1, 89), (2, 3), (3, 0), (4, 5)])
\end{lstlisting}

\section {Melakukan Reverse pada list}
\begin{lstlisting}
	array=[0,10,20,40]
	for i in reversed(array):
	print(i)
\end{lstlisting}

\section {Cara mendapatkan nilai ASCII dari suatu karakter}
\begin{lstlisting}
	>>> ord('a')
	97
	>>> chr(97)
	'a'
	>>> chr(ord('a') + 3)
	'd'
	>>>
\end{lstlisting}

\section {Melakukan pengecekan ketersediaan variabel}
Untuk memeriksa keberadaan variabel lokal:
\begin{lstlisting}
	if 'myVar' in locals():
	# myVar exists.
\end{lstlisting}
Untuk memeriksa keberadaan variabel global:
\begin{lstlisting}
	if 'myVar' in globals():
	# myVar exists.
\end{lstlisting}
Untuk memeriksa apakah suatu objek memiliki atribut:
\begin{lstlisting}
	if hasattr(obj, 'attr_name'):
	# obj.attr_name exists.
\end{lstlisting}

\section {Ekstrak nama file dari path, apa pun format os/pathnya}
\begin{lstlisting}
	import os
	print(os.path.basename(your_path))
\end{lstlisting}

\section {Mencetak exception dengan Python?}
\begin{lstlisting}
	except Exception as e: print(e)
\end{lstlisting}

\section {Cara mendapatkan waktu saat ini dengan Python}
\begin{lstlisting}
	>>> import datetime
	>>> datetime.datetime.now()
	datetime.datetime(2009, 1, 6, 15, 8, 24, 78915)
	
	>>> print(datetime.datetime.now())
	2009-01-06 15:08:24.789150
\end{lstlisting}

\section {Membuat daftar semua file direktori}
\begin{lstlisting}
	from os import listdir
	from os.path import isfile, join
	onlyfiles = [f for f in listdir(mypath) if isfile(join(mypath, f))]
\end{lstlisting}

\section {Sort dictionary berdasarkan nilai}
\begin{lstlisting}
	>>> x = {1: 2, 3: 4, 4: 3, 2: 1, 0: 0}
	>>> {k: v for k, v in sorted(x.items(), key=lambda item: item[1])}
	{0: 0, 2: 1, 1: 2, 4: 3, 3: 4}
\end{lstlisting}
atau
\begin{lstlisting}
	>>> dict(sorted(x.items(), key=lambda item: item[1]))
	{0: 0, 2: 1, 1: 2, 4: 3, 3: 4}
\end{lstlisting}

\section {Menambah key baru pada dictionary}
\begin{lstlisting}
	d = {'key': 'value'}
	print(d)  # {'key': 'value'}
	
	d['mynewkey'] = 'mynewvalue'
	
	print(d)  # {'key': 'value', 'mynewkey': 'mynewvalue'}
\end{lstlisting}

\section {Menggabungkan dua buah list}
\begin{lstlisting}
	listone = [1, 2, 3]
	listtwo = [4, 5, 6]
	
	joinedlist = listone + listtwo
\end{lstlisting}

\section {Melakukan pengecekan list kosong}
\begin{lstlisting}
	if not a:
	print("List is empty")
\end{lstlisting}

\section {Mendefinisikan array dua dimensi}
\begin{lstlisting}
	# Creates a list containing 5 lists, each of 8 items, all set to 0
	w, h = 8, 5
	Matrix = [[0 for x in range(w)] for y in range(h)] 
\end{lstlisting}
coba tambahkan nilai seperti berikut
\begin{lstlisting}
	Matrix[0][0] = 1
	Matrix[6][0] = 3 # error! range... 
	Matrix[0][6] = 3 # valid
\end{lstlisting}

\section {Convert date ke datetime pada python}
\begin{lstlisting}
	from datetime import date
	from datetime import datetime
	
	dt = datetime.combine(date.today(), datetime.min.time())
\end{lstlisting}

\section {Cara mendapatkan hostname system}
\begin{lstlisting}
	import socket
	print(socket.gethostname())
\end{lstlisting}

\section {Membuat list kosong dengan besaran yang ditentukan}
\begin{lstlisting}
	>>> l = [None] * 10
	>>> l
	[None, None, None, None, None, None, None, None, None, None]
\end{lstlisting}

\section{Menambah key baru pada dictionary}
\begin{lstlisting}
	d = {'key': 'value'}
	print(d) # {'key': 'value'}
	
	d['mynewkey'] = 'mynewvalue'
	
	print(d) # {'key': 'value',
		'mynewkey': 'mynewvalue'}
\end{lstlisting}

\section{Cara memilihi item secara acak dari daftar}
\begin{lstlisting}
	import random
	
	foo = ['battery', 'correct', 'horse', 'staple']
	secure_random = random.SystemRandom()
	print(secure_random.choice(foo))
\end{lstlisting}

\section{Mengimpor file dari folder berbeda}
\begin{lstlisting}
	import os, sys
	
	from os.path import dirname, join, abspath
	sys.path.insert(0, abspath(join(dirname(__file__), '..')))
	
	from root_folder import file_name
\end{lstlisting}

\section{Membaca JSON yang memiliki banyak items}
\begin{lstlisting}
	data JSON dari API jsonplaceholder dengan endpoint URL: https://jsonplaceholder.typicode.com/posts
	
	Buatlah program baru dengan nama list_artikel.py, kemudian isi dengan kode berikut:
	
	import json
	from urllib import request
	
	url = "https://jsonplaceholder.typicode.com/posts"
	
	# lakukan http request
	response = request.urlopen(url)
	
	# parsing data json
	data = json.loads(response.read())
	
	# gunakan perulangan untuk menampilkan data
	for i in range(len(data)):
	print(f"{i}. {data[i]['title']}")
\end{lstlisting}

\section{Cara menampilkan Input dari Keyboard}
\begin{lstlisting}
	Python sudah menyediakan fungsi input() dan raw_input() untuk mengambil inputan dari keyboard.
	
	Cara pakainya:
	
	nama_varabel = input("Sebuah Teks")
	
	# Mengambil input
	nama = raw_input("Siapa nama kamu: ")
	umur = input("Berapa umur kamu: ")
	
	# Menampilkan output
	print "Hello",nama,"umur kamu adalah",umur,"tahun"
\end{lstlisting}

\section{Cara menampilkan output}
\begin{lstlisting}
	nama = "Muhammad Fahri Ramadhan"
	print "Hello",nama
\end{lstlisting}

\section{Menggunakan string Formatting}
\begin{lstlisting}
	nama = raw_input("Inputkan nama: ")
	umur = input("Inputkan umur: ")
	tinggi = input("Inputkan tinggi badan: ")
	
	print "Hello %s, saat ini usiamu %d tahun dan tinggi badanmu %f cm" % (nama, umur, tinggi)
\end{lstlisting}

\section{Menggunakan Perulangan for}
\begin{lstlisting}
	item = ['kopi','nasi','teh','jeruk']
	
	for isi in item:
	print(isi)
\end{lstlisting}

\section{Menggunakan Perulangan while}
\begin{lstlisting}
	jawab = 'ya'
	hitung = 0
	
	while(jawab == 'ya'):
	hitung += 1
	jawab = input("Ulang lagi tidak? ")
	
	print(f"Total perulagan: {hitung}")
\end{lstlisting}

\section{Cara pakai operator relasi sama dengan}
\begin{lstlisting}
	lulus = raw_input("Apakah kamu lulus? [ya/tidak]: ")
	
	if lulus == "tidak":
	print("Kamu harus ikut ujian")
\end{lstlisting}

\section{Penggunaan If/Else}
\begin{lstlisting}
	umur = input("Berapa umur kamu: ")
	
	if umur >= 18:
	print("Kamu boleh membuat SIM")
	else:
	print("Kamu belum boleh membuat SIM")
\end{lstlisting}

\section{Membuat Program Dengan list}
\begin{lstlisting}
	# Membuat list kosong untuk menampung hobi
	hobi = []
	stop = False
	i = 0
	
	# Mengisi hobi
	while(not stop):
	hobi_baru = raw_input("Inputkan hobi yang ke-{}: ".format(i))
	hobi.append(hobi_baru)
	
	# Increment i
	i += 1
	
	tanya = raw_input("Mau isi lagi? (y/t): ")
	if(tanya == "t"): 
	stop = True
	
	# Cetak Semua Hobi
	print "=" * 10 
	print "Kamu memiliki {} hobi".format(len(hobi))
	for hb in hobi:
	print "- {}".format(hb)
\end{lstlisting}

\section{Pengambilan panjang Tuple}
\begin{lstlisting}
	# Membuat Tuple
	hari = ('Senin', 'Selasa', 'Rabu', 'Kamis', 'Jum\'at', 'Sabtu', 'Minggu')
	
	# Mengambil panjang tuple hari
	print("Jumlah hari: %d" % len(hari))
\end{lstlisting}

\section{Mengakses nilai item dari dictionary}
\begin{lstlisting}
	# Membuat Dictionary
	mahasiswa = {
		"nama": "Muhammad Fahri Ramadhan",
		"umur": 22,
		"hobi": ["coding", "membaca", "belajar"],
		"menikah": False,
		"sosmed": {
			"facebook": "fahri-r",
			"twitter": "@fahri_r"
		} 
	}
	
	# Mengakses isi dictionary
	print("Nama saya adalah %s" % mahasiswa["nama"])
	print("Twitter: %s" % mahasiswa["sosmed"]["twitter"])
\end{lstlisting}

\section{Membuat fungsi mengembalikan nilai}
\begin{lstlisting}
	def luas_persegi(sisi):
	luas = sisi * sisi
	return luas
	
	# pemanggilan fungsi
	print "Luas persegi: %d" % luas_persegi(6)
\end{lstlisting}

\section{Cara menggunakan lambda expression}
\begin{lstlisting}
	greeting = lambda name: print(f"Hello, {name}") 
\end{lstlisting}

\section{Penggunaan dari *args dan **kwargs}
\begin{lstlisting}
	# membuat fungsi dengan parameter *args
	def kirim_sms(*nomer):
	print nomer
	
	# membuat fungsi dengan parameter **kwargs
	def tulis_sms(**isi):
	print isi
	
	# Pemanggilan fungsi *args
	kirim_sms(123, 888, 4444)
	
	# pemanggilan fungsi **kwargs
	tulis_sms(tujuan=123, pesan="apa kabar")
\end{lstlisting}

\section{Parsing XML di Python}
\begin{lstlisting}
	import xml.dom.minidom as minidom
	
	def main():
	# gunakan fungsi parse() untuk me-load xml ke memori 
	# dan melakukan parsing
	doc = minidom.parse("mahasiswa.xml")
	
	# Cetak isi doc dan tag pertamanya
	print doc.nodeName
	print doc.firstChild.tagName
	
	
	if __name__ == "__main__":
	main()
\end{lstlisting}

\section{Cara baca dan Parse file CSV di Python}
\begin{lstlisting}
	import csv
	
	with open('contacts.csv') as csv_file:
	csv_reader = csv.reader(csv_file, delimiter=",")
	print(csv_reader)
	for row in csv_reader:
	print(row)
\end{lstlisting}

\section{ Menulis data dictionary ke CSV}
\begin{lstlisting}
	import csv
	
	with open('contacts.csv', mode='a') as csv_file:
	# menentukan label
	fieldnames = ['NO', 'NAMA', 'TELEPON']
	
	# membuat objek writer
	writer = csv.DictWriter(csv_file, fieldnames=fieldnames)
	
	# menulis baris ke file CSV
	writer.writeheader()
	writer.writerow({'NO': '10', 'NAMA': 'Nanda', 'TELEPON': '02109999'})
	writer.writerow({'NO': '11', 'NAMA': 'Fahriza', 'TELEPON': '02148488888'})
	
	print("Writing Done!")
\end{lstlisting}

\section{Membuat fungsi lihat data}
\begin{lstlisting}
	def show_contact():
	clear_screen()
	contacts = []
	with open(csv_filename) as csv_file:
	csv_reader = csv.reader(csv_file, delimiter=",")
	for row in csv_reader:
	contacts.append(row)
	
	if (len(contacts) > 0):
	labels = contacts.pop(0)
	print(f"{labels[0]} \t {labels[1]} \t\t {labels[2]}")
	print("-"*34)
	for data in contacts:
	print(f'{data[0]} \t {data[1]} \t {data[2]}')
	else:
	print("Tidak ada data!")
	back_to_menu()
\end{lstlisting}

\section{Menggunakan fungsi create}
\begin{lstlisting}
	def create_contact():
	clear_screen()
	with open(csv_filename, mode='a') as csv_file:
	fieldnames = ['NO', 'NAMA', 'TELEPON']
	writer = csv.DictWriter(csv_file, fieldnames=fieldnames)
	
	no = input("No urut: ")
	nama = input("Nama lengkap: ")
	telepon = input("No. Telepon: ")
	
	writer.writerow({'NO': no, 'NAMA': nama, 'TELEPON': telepon})    
	print("Berhasil disimpan!")
	
	back_to_menu()
\end{lstlisting}

\section{Membuat fungsi search}
\begin{lstlisting}
	def search_contact():
	clear_screen()
	contacts = []
	
	with open(csv_filename, mode="r") as csv_file:
	csv_reader = csv.DictReader(csv_file)
	for row in csv_reader:
	contacts.append(row)
	
	no = input("Cari berdasrakan nomer urut> ")
	
	data_found = []
	
	# mencari contact
	indeks = 0
	for data in contacts:
	if (data['NO'] == no):
	data_found = contacts[indeks]
	
	indeks = indeks + 1
	
	if len(data_found) > 0:
	print("DATA DITEMUKAN: ")
	print(f"Nama: {data_found['NAMA']}")
	print(f"Telepon: {data_found['TELEPON']}")
	else:
	print("Tidak ada data ditemukan")
	back_to_menu()
\end{lstlisting}

\section{Membuat fungsi edit}
\begin{lstlisting}
	def edit_contact():
	clear_screen()
	contacts = []
	
	with open(csv_filename, mode="r") as csv_file:
	csv_reader = csv.DictReader(csv_file)
	for row in csv_reader:
	contacts.append(row)
	
	print("NO \t NAMA \t\t TELEPON")
	print("-" * 32)
	
	for data in contacts:
	print(f"{data['NO']} \t {data['NAMA']} \t {data['TELEPON']}")
	
	print("-----------------------")
	no = input("Pilih nomer kontak> ")
	nama = input("nama baru: ")
	telepon = input("nomer telepon baru: ")
	
	# mencari contact dan mengubah datanya
	# dengan data yang baru
	indeks = 0
	for data in contacts:
	if (data['NO'] == no):
	contacts[indeks]['NAMA'] = nama
	contacts[indeks]['TELEPON'] = telepon
	indeks = indeks + 1
	
	# Menulis data baru ke file CSV (tulis ulang)
	with open(csv_filename, mode="w") as csv_file:
	fieldnames = ['NO', 'NAMA', 'TELEPON']
	writer = csv.DictWriter(csv_file, fieldnames=fieldnames)
	writer.writeheader()
	for new_data in contacts:
	writer.writerow({'NO': new_data['NO'], 'NAMA': new_data['NAMA'], 'TELEPON': new_data['TELEPON']}) 
	
	back_to_menu()
\end{lstlisting}

\section{Menggunakan fungsi delete}
\begin{lstlisting}
	def delete_contact():
	clear_screen()
	contacts = []
	
	with open(csv_filename, mode="r") as csv_file:
	csv_reader = csv.DictReader(csv_file)
	for row in csv_reader:
	contacts.append(row)
	
	print("NO \t NAMA \t\t TELEPON")
	print("-" * 32)
	
	for data in contacts:
	print(f"{data['NO']} \t {data['NAMA']} \t {data['TELEPON']}")
	
	print("-----------------------")
	no = input("Hapus nomer> ")
	
	# mencari contact dan mengubah datanya
	# dengan data yang baru
	indeks = 0
	for data in contacts:
	if (data['NO'] == no):
	contacts.remove(contacts[indeks])
	indeks = indeks + 1
	
	# Menulis data baru ke file CSV (tulis ulang)
	with open(csv_filename, mode="w") as csv_file:
	fieldnames = ['NO', 'NAMA', 'TELEPON']
	writer = csv.DictWriter(csv_file, fieldnames=fieldnames)
	writer.writeheader()
	for new_data in contacts:
	writer.writerow({'NO': new_data['NO'], 'NAMA': new_data['NAMA'], 'TELEPON': new_data['TELEPON']}) 
	
	print("Data sudah terhapus")
	back_to_menu()
\end{lstlisting}

\section{Menggunakan main loop}
\begin{lstlisting}
	if __name__ == "__main__":
	while True:
	show_menu()
\end{lstlisting}

\section{Menangani exception}
\begin{lstlisting}
	# import modul sys untuk memperoleh jenis eksepsi
	import sys
	
	lists = ['a', 0, 4]
	for each in lists:
	try:
	print("Masukan:", each)
	r = 1/int(each)
	break
	except:
	print("Upps!", sys.exc_info()[0], " terjadi.")
	print("Masukan berikutnya.")
	print()
	print("Kebalikan dari ", each, " =", r)
\end{lstlisting}

\section {Cara menghitung tanggal dan waktu :}
\begin{lstlisting}
	from datetime import datetime
	now = datetime.now()
	mm = str(now.month)
	dd = str(now.day)
	yyyy = str(now.year)
	hour = str(now.hour)
	mi = str(now.minute)
	ss = str(now.second)
	print (mm + "/" + dd + "/" + yyyy + " " + hour + ":" + mi + ":" + ss)
\end{lstlisting}

\section {Cara menghitung 2 angka yang di inputkan :}
\begin{lstlisting}
	number1 = input("Angka Pertama: ")
	number2 = input("\nANgka Kedua: ")
	
	sum = float(number1) + float(number2)
	
	print("Penjumlahan dari {0} dan {1} adalah {2}" .format(number1, number2, sum)) 
\end{lstlisting}

\section {Cara  menampilkan nama :}
\begin{lstlisting}
	# Python program showing
	# a use of raw_input()
	g = raw_input("Enter your name : ")
	print g 
\end{lstlisting}

\section {Cara  menampilkan kalkulator :}
\begin{lstlisting}
	# Program make a simple calculator that can add, subtract, multiply and divide using functions
	# This function adds two numbers
	def add(x, y):
	return x + y
	# This function subtracts two numbers
	def subtract(x, y):
	return x - y
	# This function multiplies two numbers
	def multiply(x, y):
	return x * y
	# This function divides two numbers
	def divide(x, y):
	return x / y
	print("Select operation.")
	print("1.Add")
	print("2.Subtract")
	print("3.Multiply")
	print("4.Divide")
	# Take input from the user
	choice = input("Enter choice(1/2/3/4):")
	num1 = int(input("Enter first number: "))
	num2 = int(input("Enter second number: "))
	if choice == '1':
	print(num1,"+",num2,"=", add(num1,num2))
	elif choice == '2':
	print(num1,"-",num2,"=", subtract(num1,num2))
	elif choice == '3':
	print(num1,"*",num2,"=", multiply(num1,num2))
	elif choice == '4':
	print(num1,"/",num2,"=", divide(num1,num2))
	else:
	print("Invalid input")
\end{lstlisting}

\section {Cara untuk bilangan Fibonacci : }
\begin{lstlisting}
	def Fibonacci(n):
	if n<0:
	print("Incorrect input")
	# First Fibonacci number is 0
	elif n==1:
	return 0
	# Second Fibonacci number is 1
	elif n==2:
	return 1
	else:
	return Fibonacci(n-1)+Fibonacci(n-2)
	# Driver Program
	print(Fibonacci(10)) 
\end{lstlisting}

\section {Cara untuk sorting array :}
\begin{lstlisting}
	def insertionSort(arr):
	# Traverse through 1 to len(arr)
	for i in range(1, len(arr)):
	key = arr[i]
	# Move elements of arr[0..i-1], that are
	# greater than key, to one position ahead
	# of their current position
	j = i-1
	while j >=0 and key < arr[j] :
	arr[j+1] = arr[j]
	j -= 1
	arr[j+1] = key
	
	# Driver code to test above
	arr = [12, 11, 13, 8, 4]
	insertionSort(arr)
	print ("Sorted array is:")
	for i in range(len(arr)):
	print ("%d" %arr[i]) 
\end{lstlisting}

\section {Cara untuk sorting array 2:}
\begin{lstlisting}
	test_string = "Hello World My Name Is : Joo"
	# printing original string
	print ("The original string is : " + test_string)
	# using split()
	# to count words in string
	res = len(test_string.split())
	# printing result
	print ("The number of words in string are : " + str(res)) 
\end{lstlisting}

\section {Cara Sederhana Menghitung Volume Balok}
\begin{lstlisting}
	print ("PROGRAM PYTHON MENGHITUNG VOLUME BALOK")
	p = float(input("Panjang = "))
	l = float(input("Lebar   = "))
	t = float(input("Tinggi  = "))
	
	v = p*l*t
	
	print ("Volume Balok = %0.2f" %v)
\end{lstlisting}

\section {Cara Menentukan Bilangan Ganjil Genap}
\begin{lstlisting}
	bil = int(input("Masukan Bilangan :"))
	
	if bil % 2 == 0:
	print("%d Merupakan Bilangan Genap" % bil)
	else:
	print("%d Merupakan Bilangan Ganjil" % bil)
\end{lstlisting}

\section {Cara Menentukan Nilai Indeks Mahasiswa}
\begin{lstlisting}
	print("PROGRAM PYTHON MENENTUKAN NILAI INDEKS MAHASISWA")
	tugas = float(input("\nMasukkan nilai Tugas: "))
	uts = float(input("Masukkan nilai UTS: "))
	uas = float(input("Masukkan nilai UAS: "))
	
	na =  (0.15 * uas) + (0.35 * uts) +  (0.50 * uas)
	if na >= 80:
	indeks = 'A'
	elif na >= 70:
	indeks = 'B'
	elif na >= 55:
	indeks = 'C'
	elif na >= 40:
	indeks = 'D'
	else:
	indeks = 'E'
	
	print("\nNilai Akhir  = %0.2f" % na)
	print("Nilai Indeks = %c" % indeks)
\end{lstlisting}

\section {Cara Menentukan Bilangan Terkecil dan Terbesar}
\begin{lstlisting}
	print ("PROGRAM PYTHON MENGHITUNG NILAI TERKECIL & TERBESAR SERTA NILAI RATA-RATA")
	n = int(input("\nMasukan Jumlah Bilangan = "))
	bil = []
	tot = 0
	for x in range(n):
	m=x+1
	a = int(input("Bilangan ke %d = "%m))
	bil.append(a)
	tot += bil[x]
	rata2 = tot / n
	
	print("\nBilangan Terkecil : %d" %min(bil))
	print("Bilangan Terbesar : %d" %max(bil))
	print("Nilai Rata-rata   : %0.2f" %rata2)
\end{lstlisting}

\section {Cara menggunakan Fungsi Rekursif}
\begin{lstlisting}
	def pangkat(x,y):
	if y == 0:
	return 1
	else:
	return x * pangkat(x,y-1)
	
	x = int(input("Masukan Nilai X : "))
	y = int(input("Masukan Nilai Y : "))
	
	print("%d dipangkatkan %d = %d" % (x,y,pangkat(x,y)))
\end{lstlisting}

\section {perulangan while seperti for + range}
\begin{lstlisting}
	i = 1
	
	while i <= 5:
	print(i)
	i += 1
\end{lstlisting}

\section {Menjumlahkan bilangan}
\begin{lstlisting}
	jumlah = float(bil1) + float(bil2)
\end{lstlisting}

\section {perulangan while untuk list}
\begin{lstlisting}
	listKota = ['Jakarta', 'Surabaya', 'Depok', 'Bekasi', 'Solo', 'Jogjakarta', 'Semarang', 'Makassar']
	
	# bermain index
	i = 0
	while i < len(listKota):
	print(listKota[i])
	i += 1
\end{lstlisting}

\section {perulangan while dengan inputan}
\begin{lstlisting}
	a = int(input('Masukkan bilangan ganjil lebih dari 50: '))
	
	while a % 2 != 1 or a <= 50:
	a = int(input('Salah, masukkan lagi: '))
	
	print('Benar')
\end{lstlisting}

\section {perulangan while dengan continue}
\begin{lstlisting}
	listKota = ['Jakarta', 'Surabaya', 'Depok', 'Bekasi', 'Solo', 'Jogjakarta', 'Semarang', 'Makassar']
	
	# skip jika i adalah bilangan genap
	# dan i lebih dari 0
	i = -1
	while i < len(listKota):
	i += 1
	if i % 2 == 0 and i > 0:
	print('skip')
	continue
	
	# tidak dieksekusi ketika continue dipanggil
	print(listKota[i])
\end{lstlisting}

\section {perulangan while dengan break}
\begin{lstlisting}
	listKota = [
	'Jakarta', 'Surabaya', 'Depok', 'Bekasi', 'Solo',
	'Jogjakarta', 'Semarang', 'Makassar'
	]
	
	kotaYangDicari = input('Masukkan nama kota yang dicari: ')
	
	i = 0
	while i < len(listKota):
	if listKota[i].lower() == kotaYangDicari.lower():
	print('Ketemu di index', i)
	break
	
	print('Bukan', listKota[i])
	i += 1
	
\end{lstlisting}

\section {Menaikkan Exceptions}
\begin{lstlisting}
	# Raise an instance of the Exception class itself
	
	raise Exception('Ummm... something is wrong')
	
	# Raise an instance of the RuntimeError class
	
	raise RuntimeError('Ummm... something is wrong')
	
	# Raise a custom subclass of Exception that keeps the timestamp the exception was created
	
	from datetime import datetime
	
	class SuperError(Exception):
	
	def __init__(self, message):
	
	Exception.__init__(message)
	
	self.when = datetime.now()
	
	raise SuperError('Ummm... something is wrong')
\end{lstlisting}

\section {Menelan Exception}
\begin{lstlisting}
	import json
	
	import yaml
	
	def parse_file(filename):
	
	try:
	
	return json.load(open(filename))
	
	except json.JSONDecodeError
	
	return yaml.load(open(filename))
\end{lstlisting}

\section {Klausul Finally}
\begin{lstlisting}
	def fetch_some_data():
	
	db = open_db_connection()
	
	query(db)
	
	close_db_Connection(db)
\end{lstlisting}

\section {fungsi penjumlahan}
\begin{lstlisting}
	def add(x, y):
	return x + y
\end{lstlisting}

\section {Logging}
\begin{lstlisting}
	import logging
	
	logger = logging.getLogger()
	
	def f():
	
	try:
	
	flaky_func()
	
	except Exception:
	
	logger.exception()
	
	raise
\end{lstlisting}

\section {fungsi perkalian}
\begin{lstlisting}
	def multiply(x, y):
	return x * y
\end{lstlisting}

\section {fungsi pembagian}
\begin{lstlisting}
	def divide(x, y):
	return x / y
\end{lstlisting}

\section {Error Logger}
\begin{lstlisting}
	def log_error(logger)
	
	def decorated(f):
	
	@functools.wraps(f)
	
	def wrapped(*args, **kwargs):
	
	try:
	
	return f(*args, **kwargs)
	
	except Exception as e:
	
	if logger:
	
	logger.exception(e)
	
	raise
	
	return wrapped
	
	return decorated
\end{lstlisting}

\section {Retrier}
\begin{lstlisting}
	import time
	
	import math
	def retry(tries, delay=3, backoff=2):
	
	if backoff <= 1:
	
	raise ValueError("backoff must be greater than 1")
	tries = math.floor(tries)
	
	if tries < 0:
	
	raise ValueError("tries must be 0 or greater")
	if delay <= 0:
	
	raise ValueError("delay must be greater than 0")
	def deco_retry(f):
	
	def f_retry(*args, **kwargs):
	
	mtries, mdelay = tries, delay # make mutable
	rv = f(*args, **kwargs) # first attempt
	
	while mtries > 0:
	
	if rv is True: # Done on success
	
	return True
	mtries -= 1      # consume an attempt
	
	time.sleep(mdelay) # wait...
	
	mdelay *= backoff  # make future wait longer
	rv = f(*args, **kwargs) # Try again
	return False # Ran out of tries :-(
	return f_retry # true decorator -> decorated function
	
	return deco_retry  # @retry(arg[, ...]) -> true decorator
\end{lstlisting}

\section {Cara untuk mencetak semua permutasi}
\begin{lstlisting}
	from itertools import permutations
	
	# Mendapatkan semua permutasi dari [1, 2, 3]
	perm = permutations([1, 2, 3])
	
	# Print semua permutasi
	for i in perm:
	print(i)
\end{lstlisting}

\section {Menggunakan pernyataan If else}
\begin{lstlisting}
	myDict = {1: 1, 2: 4, 3: 9}
	print("The dictionary is:", myDict)
	key = 4
	if key in myDict.keys():
	print(myDict[key])
	else:
	print("{} not a key of dictionary".format(key))
\end{lstlisting}

\section {Program Menentukan Nilai Akhir Semester}
\begin{lstlisting}
	def fungsi_total_nilai(var_harian, var_uts, var_uas) :
	var_harian = int(var_harian) * 0.3
	var_uts = int(var_uts) * 0.3
	var_uas = int(var_uas) * 0.4
	
	var_total = var_harian + var_uts + var_uas
	return var_total
\end{lstlisting}

\section {Mengubah string pada list ke int}
\begin{lstlisting}
	results = list(map(int, results))
\end{lstlisting}

\section {Unzipping files}
\begin{lstlisting}
	import zipfile
	with zipfile.ZipFile(path_to_zip_file, 'r') as zip_ref:
	zip_ref.extractall(directory_to_extract_to)
\end{lstlisting}

\section {Perbandingan string dengan case-sensitive}
\begin{lstlisting}
	string1 = 'Hello'
	string2 = 'hello'
	
	if string1.casefold() == string2.casefold():
	print("The strings are the same (case insensitive)")
	else:
	print("The strings are NOT the same (case insensitive)")
\end{lstlisting}

\section {Mendefinisikan infinite number}
\begin{lstlisting}
	import math
	test = math.inf
\end{lstlisting}

\section {Mencari rata-rata pada sebuah list}
\begin{lstlisting}
	l = [15, 18, 2, 36, 12, 78, 5, 6, 9]
	
	import statistics
	statistics.mean(l)
\end{lstlisting}

\section {Merename key pada dictionary}
\begin{lstlisting}
	mydict[k_new] = mydict.pop(k_old)
\end{lstlisting}

\section {Merename file}
\begin{lstlisting}
	import os
	
	os.rename('a.txt', 'b.kml')
\end{lstlisting}

\section {Mendapatkan nama script dengan python}
\begin{lstlisting}
	os.path.basename(__file__)
\end{lstlisting}

\section {Convert JSON string ke dict}
\begin{lstlisting}
	import json
	
	d = json.loads(j)
	print d['glossary']['title']
\end{lstlisting}

\section {Mendapatkan nomor minggu}
\begin{lstlisting}
	>>> import datetime
	>>> datetime.date(2010, 6, 16).isocalendar()[1]
	24
\end{lstlisting}

\section {Date string ke date object}
\begin{lstlisting}
	>>> import datetime
	>>> datetime.datetime.strptime('24052010', "%d%m%Y").date()
	datetime.date(2010, 5, 24)
\end{lstlisting}

\section {Membuat MD5 Checksum dari file}
\begin{lstlisting}
	import hashlib
	def md5(fname):
	hash_md5 = hashlib.md5()
	with open(fname, "rb") as f:
	for chunk in iter(lambda: f.read(4096), b""):
	hash_md5.update(chunk)
	return hash_md5.hexdigest()
\end{lstlisting}

\section {Disable log messages dari Requests library}
\begin{lstlisting}
	import logging
	
	logging.getLogger("requests").setLevel(logging.WARNING)
\end{lstlisting}

\section {Convert dict ke kwargs}
\begin{lstlisting}
	func(type='Event')
\end{lstlisting}

\section {Rename banyak file dalam direktori}
\begin{lstlisting}
	$ ls
	cheese_cheese_type.bar  cheese_cheese_type.foo
	$ python
	>>> import os
	>>> for filename in os.listdir("."):
	...  if filename.startswith("cheese_"):
	...    os.rename(filename, filename[7:])
	... 
	>>> 
	$ ls
	cheese_type.bar  cheese_type.foo
\end{lstlisting}

\section {Convert local time string ke UTC}
\begin{lstlisting}
	from datetime import datetime   
	import pytz
	
	local = pytz.timezone("America/Los_Angeles")
	naive = datetime.strptime("2001-2-3 10:11:12", "%Y-%m-%d %H:%M:%S")
	local_dt = local.localize(naive, is_dst=None)
	utc_dt = local_dt.astimezone(pytz.utc)
\end{lstlisting}

\section {Mencari dan mereplace elemen pada list}
\begin{lstlisting}
	>>> a=[1,2,3,1,3,2,1,1]
	>>> [4 if x==1 else x for x in a]
	[4, 2, 3, 4, 3, 2, 4, 4]
\end{lstlisting}

\section {Mencari irisan pada list}
\begin{lstlisting}
	>>> a = [1,2,3,4,5]
	>>> b = [1,3,5,6]
	>>> list(set(a) & set(b))
	[1, 3, 5]
\end{lstlisting}

\end{document}











\end{document}


